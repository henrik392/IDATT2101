\documentclass[12pt,a4paper]{article}
\usepackage[utf8]{inputenc}
\usepackage[norsk]{babel}
\usepackage{amsmath, amssymb, amsthm}  % For matematisk notasjon
\usepackage{algorithm, algorithmic}    % For å skrive algoritmer
\usepackage{enumitem}
\usepackage{booktabs}
\usepackage{graphicx}                  % For å inkludere bilder
\usepackage{hyperref}                  % For hyperlenker
\usepackage{pgfplots}                  % For grafer
\pgfplotsset{compat=1.16}
\hypersetup{
    colorlinks=false,
    pdfborder={0 0 0},
    }


\title{Øving n - Algoritmer og datastrukturer}
\author{Henrik Halvorsen Kvamme}
\date{\today}

\begin{document}

\begin{center}
    \includegraphics[width=0.5\textwidth]{../images/NTNU_Logo.png}
    
    \vspace{1.5em}  % Optional vertical space
    
    {\LARGE \textbf{Øving 7} \\[0.5em] \text{Algoritmer og Datastrukturer}}  % Title
    \vspace{1em}  % Optional vertical space
    
    {\large Henrik Halvorsen Kvamme}\\  % Author name
    \vspace{0.5em}  % Optional vertical space
    
    {\today}  % Date
\end{center}

\vspace{2em}

\tableofcontents

\newpage

\section{Introduksjon}
Oppgaven handler om vektede grafer og å implementere Edmonds-Karp-algoritmen for å finne maksimal flyt. I en slik graf er hver kant gitt en viss kapasitet, som er det maksimale antallet enheter som kan ``flyte'' gjennom den kanten. Målet er å finne den maksimale mengden flyt som kan gå fra en kilde til en sluk i nettverket.

\section{Teori}
Edmonds-Karp-algoritmen er en spesifikk implementering av Ford-Fulkerson metoden for å beregne maksimal flyt i en flytnettverk.
Den bruker BFS (Breadth First Search) for å finne den korteste stien i restnettverket.

Hovedideen bak Edmonds-Karp er:
\begin{enumerate}
    \item \textbf{Start med null flyt.}
    \item Mens det finnes en sti fra kilden til slu\-ket i restnettverket (bruk BFS for dette):
    \begin{enumerate}[label=\alph*)]
        \item \textbf{Finn minimum kapasitet} over den stien - dette vil være flaskehalsen.
        \item \textbf{Send flyt} langs denne stien.
        \item \textbf{Oppdater restnettverket} med den nye flyten.
    \end{enumerate}
    \item Når det ikke finnes flere stier i restnettverket, stopp. Den nåværende \textbf{flyten er maksimal}.
\end{enumerate}

\section{Resultater}
Etter å ha brukt Edmond-Karp for alle grafene fikk jeg resultatet:
\begin{table}[h]
    \centering
    \begin{tabular}{lccc}
        \toprule
        & \textbf{k} & \textbf{s} & \textbf{maks flyt} \\
        \midrule
        flytgraf1 & 0 & 7 & 10 \\
        flytgraf2 & 0 & 1 & 27 \\
        flytgraf3 & 0 & 1 & 42 \\
        flytgraf4 & 0 & 7 & 11 \\
        flytgraf5 & 0 & 7 & 90 \\
        \bottomrule
    \end{tabular}
    \caption{Tabell representasjon av resultat fra main.cpp.}
    \label{tab:my_table}
\end{table}

\section{Konklusjon}
Jeg valgte å gjøre som forrige oppgave med å implementere en klasse grafene. Den tar inn filnavn og bruker så fstream for å lese data. Klassen har en metode for å finne maks flyt, som bruker Edmonds-Karp algoritmen.

Kildekoden ligger vedlagt i main.cpp. Main funksjonen kjører flere tester for å vise at den utfører oppgaven.

\end{document}