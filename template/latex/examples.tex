\documentclass[12pt,a4paper]{article}
\usepackage[utf8]{inputenc}
\usepackage[norsk]{babel}
\usepackage{amsmath, amssymb, amsthm}  % For matematisk notasjon
\usepackage{algorithm, algorithmic}    % For å skrive algoritmer
\usepackage{graphicx}                  % For å inkludere bilder
\usepackage{hyperref}                  % For hyperlenker
\usepackage{pgfplots}                  % For grafer
\pgfplotsset{compat=1.16}

\title{Oving n - Algoritmer og Datastrukturer}
\author{Henrik Halvorsen Kvamme}
\date{\today}

\newtheorem{theorem}{Teorem}
\newtheorem{definition}{Definisjon}
\newtheorem{example}{Eksempel}

\begin{document}

\maketitle

\section{Introduksjon}
Her kommer introduksjonen din.

\section{Algoritmer}
\subsection{Eksempelalgoritme}
\begin{algorithm}
\caption{Eksempel på en algoritme}
\begin{algorithmic}
\STATE \( n \gets \text{input}() \)
\IF{ \( n < 0 \) }
\STATE \( \text{print}(``Negative tall'') \)
\ELSE
\STATE \( \text{print}(``Ikke-negative tall'') \)
\ENDIF
\end{algorithmic}
\end{algorithm}

\section{Resultater}
\subsection{Grafisk fremstilling}
% Inkluder grafer her, laget med pgfplots eller en annen pakke

\section{Konklusjon}
Her kommer konklusjonen din.

\end{document}